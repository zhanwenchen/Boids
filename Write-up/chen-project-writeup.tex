%&pdflatex
\documentclass{article}
\usepackage{amsmath}
\usepackage{graphicx}
\renewcommand{\thesubsection}{\thesection.\alph{subsection}}

\begin{document}
\title{CS250 Final Project: Boids - A Swarm Intelligence Model}
\date{December 6, 2016}
\author{Zhanwen Chen\\Vassar College}
\maketitle

\section{Flocking: The Boids Distributed Behavior Model}

\subsection{Motivation for the model}

My final project aims to implement the boids model of swarming, created by
Craig Reynolds to generate graphics of a flock of birds for animation
in motion pictures. Instead of specifying the path of individual agents, the
modeler specifies the behavior of each agent. Each bird has a specific set
of behaviors when interacting with the

% The actual thing

\section{Follow the Leader}
\subsection{Assumptions}
  \begin{enumerate}
    \item [] As the first attempt at a flocking model, I present
      the following simplifying assumptions.

    \item 2D geometric flight. Although Craig Reynolds' ACM SIGGRAPH paper proposed
      3D geometric flight, his model is intended to produce 3D graphics instead
      of researching swarming behaviors. Meanwhile, although visually
      impressive, 3D flight necessitates motions roll and pitch, in addition to
      yaw, which is equivalent to turn in 2D flight. In addition, 3D flight
      requires a local frame of reference because of its complexity, which in
      turn requires translation from local frames of reference to the global
      frame of reference. With these additional implementation complexities,
      however, an extra dimension does not enable the model to answer
      more questions on intra-flock or flock-environment interactions. Thus,
      the choice of 2D flight in place of 3D flight in boids model is an
      easy one. As a result, we use a global frame of reference and geometric
      flight.

    \item Global perception. Each bird can perceive all variables in this
      model, including other birds and the environment. When we consider the
      alternative - localized perception, where birds have a sphere of
      perception (particularly vision), global perception seems unrealistic,
      although it simplifies our base model and serves a basis for comparison.
      In subsequent models, we implement localized perception and explore
      how changes in the range of perception impacts both intra-flock dynamics
      and flock-environment interactions.

    \item Position-oriented perception. Each bird has access to other birds'
      current and last position. This implies that each bird does not have
      access to others' current velocities but calculates their velocities
      in the last iteration based on the difference between their current
      positions and their positions in the last iteration. In other words,
      birds has a memory containing two frames - the current iteration and the
      last. The birds' latent velocity perception also makes it
      difficult to predict the next position of other birds. Subsequent models
      will change the size of memory, enabling them to average past velocities,
      and thus make a smoothed prediction. We then explore how different memory
      sizes affect flocking efficiency. Lastly, to simplify our model, we assume
      birds do not perceive or calculate accelerations of others, even though
      reality suggests otherwise.

    \item A flock leader. Different flocking strategies can result in different
      flocking outcomes. Two possibilities are leader-following and
      density-following. Global perception enables leader-following because
      each bird can identify the leader among the other birds, retrieve its
      position, calculate the velocity needed to go to the leader, and actuate
      to adopt that velocity. In subsequent models, birds will attempt to
      navigate to a predicted future position of the leader in an attempt to
      speed up the process of flocking. I will compare these two different
      navigation strategies in terms of flocking speed and other dynamics.

    \item Flock leader rotation. The environment will reassign flock leaders
      in order to make the model dynamic. In the absense of external
      stimulus such as environment obstacles and food sources, a change
      in leadership is the only source of reordering the flock. It
      also serves as a convenient tool for testing.


    \item Velocity matching. Once a bird is in a flock, its goal is to match
      the velocity of its flockmates. Consistent with the global perception
      assumption in this base model, we define this goal as matching the
      average velocity of the flock. In future models with localized
      perception, we will modify this goal to matching velocity with
      known adjacent flockmates.

    \item Flock centering. If a bird is not in a flock, its goal is to
      move closer to the leader of the flock (or a center mass in subsequent
      models).

    \item Collision avoidance. In its pursuit of velocity matching and flock
      centering, a bird must not collide with another bird (which is likely
      a flockmate). Velocity matching already contributes to the fulfillment
      of this goal. Additionally, in future models, I attempt to include
      environment obstacles as well in this endeavor.

    \item The order of priority for these three agent goals is collision
      avoidance, velocity matching, and flock centering. Collision avoidance
      is a prerequisite for velocity matching and flock centering (or any
      planned motion).

    \item Goal-oriented behavior. If we treat these three behaviors as
      discrete goals?

  \end{enumerate}

\subsection{Implementation}
  \begin{enumerate}
    \item Based on considerations from the previous section, we first specify
      the bird object and the flock object. As the basic agent of this model,
      birds have properties including position and velocity. In order to
      determine state and flock membership, we also associate each bird with
      an id.

    \item Intertwined and conflicting goals. How to adjudicate conflict?
      1. Abandon lower priority goal
      2. Graded goal with composition and fitness

    \item step-by-step:
      1. $P_{leader} - P_{self}$ %% goal_3: flock centering
      2. $v_{i} = max |v|$ with new direction parallel to $P_{leader} - P_{self}$ %% goal_3: flock centering
      3. see if $v_{i}$ also satisfies velocity matching

    \item A bird first attempts to fulfill its goal -
      if ($goal_1$ and $goal_2$ and $goal_3$)
        then do all three
      elseif ($goal_1$ and $goal_2$)
        then do 1 and 2
      else
        then do 1

      $$composite_{goal} = composite_{g1} + composite_{g2} + composite_{g3}$$
      max fitness over actuation variables =
        $$root((composite_g1 - goal_1)^2$$
          $$+ (composite_g1 - goal_2)^2$$
          $$+ (composite_g1 - goal_3)^2)$$



    \item We define the physical space a bird occupies as a square.
    \item Each bird at iteration i has only its position and velocity.
      $$Bird_{i,b} = [P_{i,b}, v_{i,b}]$$.
    \item The update in each iteration is
      $$dBird_{i,b} = [dP_{i,b}, dv_{i,b}]$$.
    \item where $dv = a$.
    \item So the invariant for each bird is
      $$Bird_{i+1,b} = Bird_{i,b} + dBird_{i,b}$$, or
      $$Bird_{i,b} = Bird_{i-1,b} + dBird_{i-1,b}$$

    \item We consider implementing the three behaviors in order of precedence.

    \item Collision detection.

    \item Steering. Turning $-\frac{\pi}{2}$ (90 degrees to the right from
      $(1,0)$ to $(0,-1)$ is $dv = (-1, -1))$.

      Turning $\frac{\pi}{2}$ (90 degrees to the left from
      $(1,0)$ to $(0,1)$ is $dv = (-1, 1))$.

  \end{enumerate}


\subsection{Questions}

  \begin{enumerate}
    \item [] Corresponding to our basic assumptions for this simplistic model,
      we want to change them to extend our model.

    \item (Question) Variable speed.

    \item (Question) Localized Perception. I expect localized perception to slow down
      the speed of flocking because the subroutine of steering towards the
      center mass or the flock leader would no longer be possible without being
      able to perceive them. Instead, a process of identifying the relative
      density of neighboring agents would result in localized flocking. Based
      on the phenomenon of localized flocking, I am also interested in exploring
      the mechanism of local flocks into a bigger flock.

    \item (Question) Localized Collision Detection.

    \item (Question) Center Mass instead of flock leader

    \item ?? (Question) Dynamic Steering. Can turn a range of degrees each step.

    \item (Question) Change order of behaviors: (permutations of order).

    \item (Question) Position perception instead of velocity perception.

    \item Environment-flock/-bird interactions: Obstacles?

    \item What am I measuring: flock size? Num of iterations it takes to achieve a full flock?

    \item What is a flock: flock membership based on distance proximity between birds?
      velocity proximity between birds?

    \item
      Affecting flocking behavior (speed to achieve full flock):
      1. changing flocking strategy - center mass vs leader
      2. interactions with the environment (obstacle, food)

  \end{enumerate}



% Bibliography Section
\begin{thebibliography}{9}
\bibitem{reynoldsarticle}
Flocks, herds and schools: A distributed behavioral model,
\\\texttt{http://dl.acm.org/citation.cfm?id=37406}
\end{thebibliography}

\end{document}
