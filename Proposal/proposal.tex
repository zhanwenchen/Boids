%&pdflatex
\documentclass{article}
\usepackage{amsmath}
\usepackage{graphicx}
\renewcommand{\thesubsection}{\thesection.\alph{subsection}}

\begin{document}
\title{CS250 Final Project Proposal: Boids - A Swarm Intelligence Model}
\date{November 23, 2016}
\author{Zhanwen Chen\\Vassar College}
\maketitle

\section{The Boids Distributed Behavior Model}

\subsection{Motivation for the model}

My final project aims to implement the boids model of swarming, created by
Craig Reynolds to generate graphics of a flock of birds for animation
in motion pictures. Instead of specifying the path of individual agents, the
modeler specifies the behavior of each agent.

\subsection{The Boids Model}

Components in the boids model include

{\begin{enumerate}
  \item [] Movement. The mechanism of movement in the boids model is 3D geometric flight.
  In terms of coordinates, the model contains both global coordinates and local
  coordinates (perception and action are localized). In terms of actions, the
  basic operationsthe include roll (rotate around the z-axis, used for banking),
  pitch (rotate around the x-axis), and yaw (rotate around the y-axis).
  \item [] Behaviors. The three basic behaviors specified by the boids model are collision
  avoidance, velocity matching, and flock centering, in order of precedence.
  Collision avoidance and velocity matching both have an effect of avoiding
  collision with flocking mates. Flock centering means an agent moves toward the
  centroid of nearby flockmates. In addition, the acceleration of the flock is the
  average of agents' acceleration.
  \item [] Localized Perception. In the boids model, an agent's environmental variables
  include the position, shape, and velocity (and even intention, see subsction
  Deliberation, Intentions, and Emotions) of nearby flockmates within the agent's
  sphere of vision.
  \item [] Obstacle evasion. A crucial feature of the flock is group collision evation
  of environment obstacles. Use steer-to-avoid to localize this behavior.
\end{enumerate}}

\section{Exploring and Extending the Boids Model}

\subsection{Flying Between Skyscrapers}

A basic task for a flock is navigating metropolitan landscapes. Implement the
basic model and move the flock along city streets. Assume skyscrapers are
infinitely tall.

\subsection{Deliberation, Intentions, and Emotions}

The existing boids model is reactive, using a set of predetermined instructions
to plan the next action. Use "emotions" with Markov Chains to implement
the Belief-Desire-Intention behavioral model.

% Bibliography Section
\begin{thebibliography}{9}
\bibitem{reynoldsarticle}
Flocks, herds and schools: A distributed behavioral model,
\\\texttt{http://dl.acm.org/citation.cfm?id=37406}
\end{thebibliography}

\end{document}
